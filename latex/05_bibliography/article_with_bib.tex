\documentclass{article}

% Language setting
\usepackage[english]{babel}

% Set page size and margins
\usepackage[letterpaper,top=2cm,bottom=2cm,left=3cm,right=3cm,marginparwidth=1.75cm]{geometry}

% Useful packages
\usepackage{amsmath}
\usepackage{graphicx}
\usepackage[colorlinks=true, allcolors=blue]{hyperref}

\title{Your Paper}
\author{You}

\begin{document}
\maketitle

\begin{abstract}
This is a complete article template demonstrating citations and bibliography.
We discuss the Capital Asset Pricing Model and the Fama-French factors.
\end{abstract}

\section{Introduction}

The Capital Asset Pricing Model (CAPM), developed by \cite{sharpe1964capital},
provides a framework for understanding the relationship between risk and
expected return. This model laid the groundwork for decades of research
in asset pricing.

\section{The Fama-French Model}

Building on the CAPM, \cite{fama1992cross} demonstrated that size and
book-to-market factors explain a significant portion of cross-sectional
variation in stock returns beyond what beta alone captures.

\section{Mathematical Framework}

The CAPM states that the expected return of an asset is:
\begin{equation}
    E[R_i] = R_f + \beta_i (E[R_m] - R_f)
    \label{eq:capm}
\end{equation}
where $R_f$ is the risk-free rate, $R_m$ is the market return, and $\beta_i$
measures the asset's sensitivity to market movements.

The Fama-French three-factor model extends this to:
\[
    E[R_i] - R_f = \beta_i^{MKT}(R_m - R_f) + \beta_i^{SMB} \cdot SMB + \beta_i^{HML} \cdot HML
\]

\section{Conclusion}

Both \cite{sharpe1964capital} and \cite{fama1992cross} represent foundational
work in understanding asset returns.

\bibliographystyle{alpha}
\bibliography{bibliography}

\end{document}
