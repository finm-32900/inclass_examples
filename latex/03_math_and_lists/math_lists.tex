\documentclass{article}

\usepackage[english]{babel}
\usepackage[letterpaper,top=2cm,bottom=2cm,left=3cm,right=3cm]{geometry}
\usepackage{amsmath}
\usepackage{amssymb}

\title{Mathematics and Lists}
\author{Your Name}

\begin{document}

\maketitle

\section{Inline Mathematics}

LaTeX excels at typesetting mathematics. Use dollar signs for inline math: $y = mx + b$.
Greek letters are easy: $\alpha$, $\beta$, $\gamma$, $\sigma$, $\mu$.

\section{Display Mathematics}

For standalone equations, use backslash-brackets:
\[
    x = \frac{-b \pm \sqrt{b^2 - 4ac}}{2a}.
\]
Or use the equation environment for numbered equations:
\begin{equation}
    S_n = \frac{1}{n} \sum_{i=1}^{n} X_i
    \label{eq:sample_mean}
\end{equation}
We can reference Equation~\ref{eq:sample_mean} later in the text.

\subsection{Aligned Equations}

The \texttt{align} environment is used for multi-line equations with alignment points (marked by \texttt{\&}):
\begin{align}
    f(x) &= x^2 + 2x + 1 \label{eq:first} \\
    &= (x + 1)^2 \label{eq:second}
\end{align}
Each line gets its own equation number. We can reference them: Equation~\ref{eq:first} and Equation~\ref{eq:second}.

The starred version \texttt{align*} suppresses all equation numbers:
\begin{align*}
    \mathbb{E}[X] &= \mu \\
    \text{Var}(X) &= \mathbb{E}[(X - \mu)^2] \\
    &= \mathbb{E}[X^2] - \mu^2
\end{align*}

\textbf{The difference:} The \texttt{*} in \texttt{align*} means ``no numbering.'' This convention is consistent across many \LaTeX{} environments---adding \texttt{*} removes automatic numbering (e.g., \texttt{equation*}, \texttt{gather*}).

\section{Common Math Notation}

Subscripts and superscripts: $x_i$, $x^2$, $x_i^2$, $e^{-rt}$

Fractions: $\frac{a}{b}$, $\frac{\partial f}{\partial x}$

Square roots: $\sqrt{2}$, $\sqrt[3]{8}$

Summations and products:
\[
\sum_{i=1}^{n} x_i \quad \prod_{j=1}^{m} y_j \quad \int_0^\infty e^{-x} dx
\]

\section{Lists}

\subsection{Numbered Lists}

\begin{enumerate}
    \item First item
    \item Second item
    \item Third item
\end{enumerate}

\subsection{Bullet Points}

\begin{itemize}
    \item Apples
    \item Oranges
    \item Bananas
\end{itemize}

\section{Text Formatting}

You can make text \textbf{bold}, \textit{italic}, or \texttt{monospace}.

Use \emph{emphasis} for semantic markup---it's italic in regular text
but upright in italic context.

\end{document}
